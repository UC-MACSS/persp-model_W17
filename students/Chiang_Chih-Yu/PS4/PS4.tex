\documentclass[letterpaper,12pt]{article}
\usepackage{array}
\usepackage{threeparttable}
\usepackage{geometry}
\geometry{letterpaper,tmargin=1in,bmargin=1in,lmargin=1.25in,rmargin=1.25in}
\usepackage{fancyhdr,lastpage}
\pagestyle{fancy}
\lhead{}
\chead{}
\rhead{}
\lfoot{}
\cfoot{}
\rfoot{\footnotesize\textsl{Page \thepage\ of \pageref{LastPage}}}
\renewcommand\headrulewidth{0pt}
\renewcommand\footrulewidth{0pt}
\usepackage[format=hang,font=normalsize,labelfont=bf]{caption}
\usepackage{listings}
\lstset{frame=single,
  language=Python,
  showstringspaces=false,
  columns=flexible,
  basicstyle={\small\ttfamily},
  numbers=none,
  breaklines=true,
  breakatwhitespace=true
  tabsize=3
}
\usepackage{amsmath}
\usepackage{amssymb}
\usepackage{amsthm}
\usepackage{harvard}
\usepackage{setspace}
\usepackage{float,color}
\usepackage[pdftex]{graphicx}
\usepackage{hyperref}
\hypersetup{colorlinks,linkcolor=red,urlcolor=blue}
\theoremstyle{definition}
\newtheorem{theorem}{Theorem}
\newtheorem{acknowledgement}[theorem]{Acknowledgement}
\newtheorem{algorithm}[theorem]{Algorithm}
\newtheorem{axiom}[theorem]{Axiom}
\newtheorem{case}[theorem]{Case}
\newtheorem{claim}[theorem]{Claim}
\newtheorem{conclusion}[theorem]{Conclusion}
\newtheorem{condition}[theorem]{Condition}
\newtheorem{conjecture}[theorem]{Conjecture}
\newtheorem{corollary}[theorem]{Corollary}
\newtheorem{criterion}[theorem]{Criterion}
\newtheorem{definition}[theorem]{Definition}
\newtheorem{derivation}{Derivation} % Number derivations on their own
\newtheorem{example}[theorem]{Example}
\newtheorem{exercise}[theorem]{Exercise}
\newtheorem{lemma}[theorem]{Lemma}
\newtheorem{notation}[theorem]{Notation}
\newtheorem{problem}[theorem]{Problem}
\newtheorem{proposition}{Proposition} % Number propositions on their own
\newtheorem{remark}[theorem]{Remark}
\newtheorem{solution}[theorem]{Solution}
\newtheorem{summary}[theorem]{Summary}
%\numberwithin{equation}{section}
\bibliographystyle{aer}
\newcommand\ve{\varepsilon}
\newcommand\boldline{\arrayrulewidth{1pt}\hline}
\begin{document}

% ----------------------------------------------------------------------
\begin{flushleft}
  \textbf{\large{Problem Set \#4}} \\
  MACS 30100, Dr. Evans \\
  Chih-Yu Chiang \\
  Python Version: 3.5.2
\end{flushleft}
\vspace{5mm}
\noindent\textbf{Problem 1} \\
\\
The minimization processes in the 2 estimations of (c) and (d) both succeeded.
For acquiring converged results from the algorithms, the estimation settings are as follow: \\
(c). error simple mode = False; minimization method = L-BFGS-B. \\
(d). error simple mode = False; minimization method = TNC. \\

\noindent\textbf{Part (a). histogram} \\
\\
A histogram of annual incomes of students who graduated in 2018, 2019, and 2020 from the University of Chicago M.A. Program in Computational Social Science. \\
\begin{figure}[htb]\centering\captionsetup{width=6.0in}
  \caption{\textbf{}}
  \fbox{\resizebox{4.0in}{3.0in}{\includegraphics{1a.png}}}
\end{figure} \\

\noindent\textbf{Part (b). lognormal pdf function} \\
\\
With $\mu$ = 5.0 and $\sigma$ = 1.0, and xvals = np.array([[200.0, 270.0], [180.0, 195.5]], \\
The lognormal PDF values are [[ 0.0019079   0.00123533] [ 0.00217547  0.0019646 ]].
\\


\clearpage

% ----------------------------------------------------------------------
\noindent\textbf{Part (c). One step SMM with mean and std} \\
\\
One step SMM is estimated with mean and standard deviation as moments and identity weighting matrix. The initial guess of $\mu$ is 11 and $\sigma$ is 0.2, with 300 simulations and 200 observations each.\\
The criterion value is 7.20176828929e-14. The estimated parameters are as follows:
\[\mu_{SMM1}= 11.3307149892\]
\[\sigma_{SMM1}= 0.208868329657\]

\begin{center}
\begin{tabular}{ c|c|c }
 moments & mean & $std^2$ \\
 \hline
 data & 85276.8236063 & 323731572.23 \\
 model & 85276.8464227 & 323731565.513 \\
 (data-model)/data & -2.67557875682e-07 & 2.07476758571e-08
\end{tabular}
\end{center}
\\

A lognormal pdf with estimated parameters is plotted. \\

\begin{figure}[htb]\centering\captionsetup{width=6.0in}
  \caption{\textbf{}}
  \fbox{\resizebox{4.in}{3.0in}{\includegraphics{1c.png}}}
\end{figure} \\

\clearpage

% ----------------------------------------------------------------------
\noindent\textbf{Part (d). Two step SMM with mean and std} \\
\\
Two step SMM is estimated with mean and standard deviation as moments and weighting matrix derived from (c). The initial guess of $\mu$ and $\sigma$ is the estimated results from (c). \\
The criterion value is 0.0120239766133. The estimated parameters are as follows:
\[\mu_{SMM2}= 11.3307146745\]
\[\sigma_{SMM2}= 0.208868331657\]

\begin{center}
\begin{tabular}{ c|c|c }
 moments & mean & $std^2$ \\
 \hline
 data & 85276.8236063 & 323731572.23 \\
 model & 85276.8196196 & 323731368.346 \\
 (data-model)/data & 4.67498781548e-08 & 6.2979110409e-07
\end{tabular}
\end{center}
\\

Lognormal pdfs with estimated parameters in previous and this questions are plotted. \\

\begin{figure}[htb]\centering\captionsetup{width=6.0in}
  \caption{\textbf{}}
  \fbox{\resizebox{4.0in}{3.0in}{\includegraphics{1d.png}}}
\end{figure} \\


\end{document}
