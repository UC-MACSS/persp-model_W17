\documentclass[letterpaper,12pt]{article}
\usepackage{array}
\usepackage{threeparttable}
\usepackage{geometry}
\geometry{letterpaper,tmargin=1in,bmargin=1in,lmargin=1.25in,rmargin=1.25in}
\usepackage{fancyhdr,lastpage}
\pagestyle{fancy}
\lhead{}
\chead{}
\rhead{}
\lfoot{}
\cfoot{}
\rfoot{\footnotesize\textsl{Page \thepage\ of \pageref{LastPage}}}
\renewcommand\headrulewidth{0pt}
\renewcommand\footrulewidth{0pt}
\usepackage[format=hang,font=normalsize,labelfont=bf]{caption}
\usepackage{listings}
\lstset{frame=single,
  language=Python,
  showstringspaces=false,
  columns=flexible,
  basicstyle={\small\ttfamily},
  numbers=none,
  breaklines=true,
  breakatwhitespace=true
  tabsize=3
}
\usepackage{amsmath}
\usepackage{amssymb}
\usepackage{amsthm}
\usepackage{harvard}
\usepackage{setspace}
\usepackage{float,color}
\usepackage[pdftex]{graphicx}
\usepackage{hyperref}
\hypersetup{colorlinks,linkcolor=red,urlcolor=blue}
\theoremstyle{definition}
\newtheorem{theorem}{Theorem}
\newtheorem{acknowledgement}[theorem]{Acknowledgement}
\newtheorem{algorithm}[theorem]{Algorithm}
\newtheorem{axiom}[theorem]{Axiom}
\newtheorem{case}[theorem]{Case}
\newtheorem{claim}[theorem]{Claim}
\newtheorem{conclusion}[theorem]{Conclusion}
\newtheorem{condition}[theorem]{Condition}
\newtheorem{conjecture}[theorem]{Conjecture}
\newtheorem{corollary}[theorem]{Corollary}
\newtheorem{criterion}[theorem]{Criterion}
\newtheorem{definition}[theorem]{Definition}
\newtheorem{derivation}{Derivation} % Number derivations on their own
\newtheorem{example}[theorem]{Example}
\newtheorem{exercise}[theorem]{Exercise}
\newtheorem{lemma}[theorem]{Lemma}
\newtheorem{notation}[theorem]{Notation}
\newtheorem{problem}[theorem]{Problem}
\newtheorem{proposition}{Proposition} % Number propositions on their own
\newtheorem{remark}[theorem]{Remark}
\newtheorem{solution}[theorem]{Solution}
\newtheorem{summary}[theorem]{Summary}
%\numberwithin{equation}{section}
\bibliographystyle{aer}
\newcommand\ve{\varepsilon}
\newcommand\boldline{\arrayrulewidth{1pt}\hline}


\begin{document}

\begin{flushleft}
  \textbf{\large{Problem Set \#2}} \\
  Perspectives on Computational Modeling \\
  MACS 30100, Dr. Evans \\
  HyungJin Cho
\end{flushleft}

\vspace{5mm}

\begin{enumerate}
  \textbf{Problem 1.}
\end {enumerate}
\begin{enumerate}
  \textbf{Part (a). Histogram}
\par
\begin{figure}[H]\centering\captionsetup{width=4.0in}
   \fbox{\resizebox{4.0in}{3.0in}{\includegraphics{./images/1(a).png}}}
\end{figure}
\par\bigskip
\end {enumerate}

\begin{enumerate}
  \textbf{Part (b). One step GMM}
\par
\begin{figure}[H]\centering\captionsetup{width=4.0in}
  \fbox{\resizebox{4.0in}{3.0in}{\includegraphics{./images/1(b).png}}}
\end{figure}
\par
GMM lognormal parameters: mu = 11.3369, sig = 0.2130 \\
Data moment: mu = 85276.8236, std = 17992.5421 \\
Model moment: mu = 85276.7904, std = 17992.5391 \\
Value of GMM criterion: 1.794297513712258e-13 \\
\par\bigskip
\end {enumerate}

\begin{enumerate}
  \textbf{Part (c). Two step GMM}
\par
\begin{figure}[H]\centering\captionsetup{width=4.0in}
  \fbox{\resizebox{4.0in}{3.0in}{\includegraphics{./images/1(c).png}}}
\end{figure}
\par
GMM lognormal parameters: mu = 11.0857, sig = 0.7706 \\
Model moment: mu = 85276.8236, std = 17992.5421 \\
Value of GMM criterion: 0.009984728414565325 \\
\par\bigskip
\end {enumerate}

\begin{enumerate}
  \textbf{Part (d). One step GMM}
\par
\begin{figure}[H]\centering\captionsetup{width=4.0in}
  \fbox{\resizebox{4.0in}{3.0in}{\includegraphics{./images/1(d).png}}}
\end{figure}
\par
Data moment: \\
The proportion of income less than $75,000 = 0.3 \\
The proportion of income between $75,000 and $100,000 = 0.5 \\
The proportion of income more than $100,000 = 0.2 \\
Model Moment: \\
The proportion of income less than $75,000 = 0.3000 \\
The proportion of income between $75,000 and $100,000 = 0.5000 \\
The proportion of income more than $100,000 = 0.2000 \\
Value of GMM criterion: 2.534788361602213e-11
\par\bigskip
\end {enumerate}

\begin{enumerate}
  \textbf{Part (e). Two step GMM}
\par
\begin{figure}[H]\centering\captionsetup{width=4.0in}
  \fbox{\resizebox{4.0in}{3.0in}{\includegraphics{./images/1(e).png}}}
\end{figure}
\par
Model Moment: \\
The proportion of income less than $75,000 = 0.2931 \\
The proportion of income between $75,000 and $100,000 = 0.5073 \\
The proportion of income more than $100,000 = 0.1996 \\
Value of GMM criterion: 91.95874552987516
\par\bigskip
\end {enumerate}

\begin{enumerate}
  \textbf{Part (f). Estimation comparison} \\
\par\bigskip
The PDF generated from 2-step GMM with 3 data moments(Part.(e)) best fits\\
the actual data and the model moments from 2-step GMM with mean and std \\ (Part.(c)) least fits the data. Other five figures also fits the data well. \\
\par\bigskip
\end{enumerate}

\begin{enumerate}
  \textbf{Problem 2.}
\end {enumerate}

\begin{enumerate}
  \textbf{Part (a). Linear regression} \\
\par\bigskip
$\beta_{0}$ = 0:252 \\
$\beta_{1}$ = 0.013 \\
$\beta_{2}$ = 0.401 \\
$\beta_{3}$ = -0.010 \\
Value of GMM criterion: 0.001821 \\
\par\bigskip
\end {enumerate}

\end{document}
